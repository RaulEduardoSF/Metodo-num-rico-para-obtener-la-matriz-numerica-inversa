\documentclass[a4paper]{scrartcl}
\usepackage[utf8]{inputenc}
\usepackage{fullpage}
\usepackage{tasks}
\usepackage{amsmath,graphicx,amssymb,amsfonts,mathtools}
\usepackage[spanish]{babel}
%\usepackage[T1]{fontenc}
%\usepackage{fourier}
\usepackage[dvipsnames]{xcolor}
\usepackage{soul}
\usepackage{alltt}
\newcommand{\then}{\Rightarrow}
\definecolor{blue20}{rgb}{0.9,0.85,1}
\sethlcolor{blue20}
%\pagecolor{black}

\begin{document}
\begin{center}
    \huge{Universidad Autonóma de Nuevo León} \\
    \LARGE{Facultad de Ciencias Fisico Matemáticas}
\end{center}

\vspace{1 cm}
\begin{center}
    \Large{Metodos Matemáticos de la Fisica II}
\end{center}

\vspace{0.5cm}
\begin{center}
    \LARGE\textbf{Metodo numérico para obtener la matriz numerica inversa}\\
\end{center}

\vspace{1.5cm}
\begin{center}
    \large Lissette Galvan Tlapale 
\end{center}
\begin{center}
    \large Raul Eduardo Santoy Flores 
\end{center}
\begin{center}
    \large Fernando Martinez Diaz
\end{center}

\vspace{6cm}
\begin{center}
    \today
\end{center}

\newpage
\section{Introducción}

Los calculos operacionales de matrices cuadradas por lo regular suelen complicarse cuando se incrementa su dimensión. Recurrimos a solucionarlo numericamente implementando el metodo de la \textbf{adjunta} para obtener la matriz inversa y la \textbf{regla de Cramer} para obtener los determinantes de todas las matrices obtenidas. Escribimos el codigo en fortran.

\section{Creando la matriz de $9\times9$}

Construimos la matriz $A$ de dimensión $n=9$ con los siguientes ciclos
\begin{verbatim}
      n = 9
      Do i = 1,n
         A(i,i) = 4
      end do

      Do i =1,n
         if (mod(i,3).ne.0.and.i.lt.9) then
            A(i,i+1) = -1
         end if
         if (mod(i-1,3).ne.0.and.i.gt.1) then
            A(i,i-1) = -1
         end if
         if (i.lt.7) then
            A(i,i+3) = -1
         end if
         if (i.gt.3) then
            A(i,i-3) = -1
         end if
      end do
\end{verbatim}
Como resultado obtenemos
\begin{figure}[h!]
    \centering
    \includegraphics{Matriz A.png}
    \caption{Contruyendo matriz $A$ a partir de ciclos}
    \label{Matriz A}
\end{figure}

Construimos nuestro vector $Y(i)$
\begin{verbatim}
      Y(1) = 75
      Y(2) = 0
      Y(3) = 50
      Y(4) = 75
      Y(5) = 0
      Y(6) = 50
      Y(7) = 175
      Y(8) = 100
      Y(9) = 150
\end{verbatim}
Lo que sigue es aplicar el metodo de la adjunta para obtener la matriz inversa $M$, donde muchas de las variables son contadoras. Calculamos el determinante de $A$ utilizando la subrutina \texttt{TSuperior(n,A,b)} \begin{alltt}
      \textcolor{blue}{Call TSuperior(n,A,b)}
      DetA = (-1)**b
      !Producto de la diagonal
      Do i = 1, n
         DetA = DetA*A(i,i)
      end do
      Write(*,*) ' Det(A) =',DetA
\end{alltt}

Esta subrutina aprovecha las propiedades de los determinantes para reducir a una matriz triangular superior. La finalidad de esto es poder calcular el determinante de cualquier matriz cuadrada como el producto de los elementos en la diagonal de la matriz trangular superior que produce \texttt{TSuperiro}. La variable \texttt{b} es un contador de intercambio de renglones y \texttt{n} es la dimensión de la matriz a reducir. Las demas variales son contadoras o auxiliares.
\begin{alltt}
      !________________Reducción a matriz triangular superior___________
      Subroutine TSuperior(d,M,b)

      Real M,C,R
      Integer i,k,t,d,b
      Dimension M(10,10),C(10),R(10)

      b = 0
      Do i = 1, d - 1

         !Intercambio de renglones
         k = 0
         Do While (M(i,i).eq.0.AND.i.lt.d.AND.k.lt.d-i)
            k = k + 1
            if (M(i+k,i).ne.0) then
               Do t = 1, d
                  R(t) = M(i,t)
                  M(i,t) = M(i+k,t)
                  M(i+k,t) = R(t)
               end do
               b = b + 1
            end if
         end do

         !Vector columna
         Do k =  1, d
            C(k) = M(k,i)
         end do

         !Operación entre renglones (+)(-)
         Do k = i + 1, d
            t = 1
            !Proceso por eliminacion por el menor en el vector i
            Do While (C(k).ne.0.AND.t.le.d)
               M(k,t) = M(k,t) - (C(k)/C(i))*M(i,t)
               t = t + 1
            end do
         end do

      end do

      Return
      End Subroutine
\end{alltt}
\hline
\vspace{0.6cm}

El determinante de $A$ tiene que ser distinto de cero para poder aplicar construir la matriz adjunta \texttt{Adj}. Iterativamente se vuelve a llamar a la subrutina $TSuperior$ para reduir las matrices obtenidas en el procedimiento y poder calcular su determinante. 
\begin{alltt}
      if (DetA.ne.0) then

         Do i = 1, n
            Do j = 1, n

              !Construyendo el menor M(i,j)
               u = 1
               Do k = 1, n
                  v = 0
                  Do t = 1, n
                     if (k.ne.i.AND.t.ne.j) then
                        v = v + 1
                        M(u,v) = A(k,t)
                     end if
                  end do
                  if (v.eq.n-1) then
                     u = u + 1
                  end if
               end do

               d = n - 1

               !Determinante del menor M(i,j)
               if (n.gt.2) then
                  \textcolor{blue}{Call TSuperior(d,M,b)}
               end if
               Det(i,j) = (-1)**b
               Do k = 1, n - 1
                  Det(i,j) = Det(i,j)*M(k,k)
               end do

               !Elemento (i,j) de la matriz adjunta
               Adj(i,j) = (-1)**(i + j)*Det(i,j)

            end do
         end do
\end{alltt}
Por ultimo, se construye la matriz inversa \texttt{IA}
\begin{alltt}
         !Matriz inversa
         Do i = 1, n
            Do j = 1, n
               IA(i,j) = Adj(j,i)/DetA
            end do
         end do
         Write(*,*) ' Matriz Inversa de A'
         Write(*,*) ' '
         Do i = 1, n
            Write(*,99) (IA(i,j), j = 1, n)
         end do
  99     Format(9f8.5,9X)
\end{alltt}
\begin{figure}[h!]
    \centering
    \includegraphics[scale = 0.7]{Inversa A.png}
    \caption{Matriz Inversa de A}
    \label{Matriz Inversa de $A$}
\end{figure}
\begin{alltt}

         Write(*,*) ' '
         Write(*,*) 'Soluciones T(i,j) expresada en vector'
         Write(*,*) ' '
         Do i = 1, n
            T(i) = 0
            Do j = 1, n
               T(i) = T(i) +  IA(i,j)*Y(j)
            end do
            Write(*,*) 'T(',i,') =',X(i)
         end do
         
      end if

      PAUSE
      END PROGRAM
\end{alltt}
Obteniendo como resultados
\begin{figure}[h!]
    \centering
    \includegraphics{Soluciones.png}
    \caption{Soluciones $T_{i,i}$}
\end{figure}
\end{document}
